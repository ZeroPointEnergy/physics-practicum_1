\documentclass[a4paper,10pt]{article}
\usepackage[latin1]{inputenc}
%\usepackage{color}
%\usepackage{colortbl}
%\usepackage{amstext}
\usepackage{amsmath}
%\usepackage{graphicx}
\usepackage{pgfplotstable}

\makeatletter
\makeatother

\begin{document}

\title{Transversalschwingung einer gespannten Saite}
\author{Andreas Zuber, Florian Schnider}

\maketitle

\section{Einleitung}

\subsection{Ziel des Versuches}

\subsection{Versuchsaufbau}\label{sec:versuchsaufbau}
% - Skizze
% - Ger�tenummern
% ...

\section{Theorie}\label{sec:theorie}
% - Zusammenfassung und Zusammenstellung der f�r die Auswertungben�tigten Formeln

\section{Fehlerrechnung}

\section{Praktische Aufgaben}
% - Alle Messwerte in �bersichtlichen Tabellen mit Mittelwerten und Fehlerangaben

\section{Messung}

\begin{itemize}
 \item K [N] : Zugkraft
 \item f [Hz] : Frequenz
 \item n [-] : Anzahl Knoten
\end{itemize}


\begin{center}
\pgfplotstabletypeset[
  col sep=comma,
  every head row/.style={before row=\hline,after row=\hline\hline},
  every last row/.style={after row=\hline},
  every first column/.style={
    column type/.add={|}{}
  },
  every last column/.style={
    column type/.add={}{|}
  }
]{versuch_05_saite_messung_01.csv}
\end{center}


\section{Diskussion}
% - Kommentare
% - Vergleich der Messwerte mit Theorie und Literaturwerten
% - Hinweis auf m�glich Fehlerquellen (besonders systematische Fehler)
% - Schwierigkeiten bei der Messung
% ...

\begin{thebibliography}{99}

\bibitem{Script} Anleitung zum Physikpraktikum, Praktikumsskript FS 2012, P. Wurz

\end{thebibliography}

\end{document}