\documentclass[a4paper,10pt]{article}
\usepackage[latin1]{inputenc}
%\usepackage{color}
%\usepackage{colortbl}
%\usepackage{amstext}
\usepackage{amsmath}
%\usepackage{graphicx}
\usepackage{pgfplotstable}

\makeatletter
\makeatother

\begin{document}

\title{Molare W�rmekapazit�t}
\author{Andreas Zuber, Florian Schnider}

\maketitle

\section{Einleitung}

\subsection{Versuchsaufbau}\label{sec:versuchsaufbau}
% - Skizze
% - Ger�tenummern
% ...
%\begin{center}
 %\includegraphics[scale=0.4]{./versuch_09_prisma_aufbau.jpeg}
 % versuch_07_windkanal_aufbau.jpeg: 1016x688 pixel, 84dpi, 30.72x20.80 cm, bb=0 0 871 590
%\end{center}

\subsection{Material}
\begin{itemize}
 \item Apparatur Nr. 111 bestehend aus Flasche und Rohr
 \item Stahlkugel
 \item Gasflaschen mit Stickstoff ($N^2$), Kohlendioxid ($CO^2$) und Argon ($Ar$)
\end{itemize}

\section{Theorie}\label{theorie}
% - Zusammenfassung und Zusammenstellung der f�r die Auswertungben�tigten Formeln

\begin{itemize}
 \item bli
\end{itemize}


\section{Praktische Aufgaben}
% - Alle Messwerte in �bersichtlichen Tabellen mit Mittelwerten und Fehlerangaben

\begin{itemize}
 \item bli
 \item bla
\end{itemize}


\begin{center}
\noindent blub:\\
\begin{tabular}{|c|c|c|c|c|}
\hline 

\hline
\end{tabular}
\end{center} 


\section{Diskussion}
% - Kommentare
% - Vergleich der Messwerte mit Theorie und Literaturwerten
% - Hinweis auf m�glich Fehlerquellen (besonders systematische Fehler)
% - Schwierigkeiten bei der Messung
% ...


\begin{thebibliography}{99}

\bibitem{Script} Anleitung zum Physikpraktikum, Praktikumsskript FS 2012, P. Wurz

\end{thebibliography}

\end{document}