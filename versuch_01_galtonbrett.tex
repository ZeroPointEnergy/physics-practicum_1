\documentclass[a4paper,10pt]{article}
\usepackage[latin1]{inputenc}
%\usepackage{color}
%\usepackage{colortbl}
%\usepackage{amstext}
\usepackage{amsmath}
%\usepackage{graphicx}
\usepackage{pgfplotstable}

\makeatletter
\makeatother

\begin{document}

\title{Einf�hrung ins experimentelle Arbeiten}
\author{Andreas Zuber, Florian Schnider}

\maketitle

\section{Einleitung}
In diesem Versuch sollen erste Erfahrungen mit statistischen Verteilungen und den Auswertungsverfahren
gesammelt werden.

\subsection{Ziel des Versuches}
Wir untersuchen die Verteilung von Kugeln in den 15 F�chern eines Galton-Bretts. Dies ist ein schr�ggestelltes 
Brett mit 14 Reihen Stiften, in exakt gleichen Abst�nden. Wenn wir nun zuoberst eine Kugel loslassen wird diese
bei jedem Stift entweder nach links oder rechts hinunter rollen. Hat die Kugel eines der F�cher erreicht hat
sie auf ihrem Weg 14 Stifte passiert also 14 mal entweder links oder rechts runter gerollt.

\section{Theorie}
% - Zusammenfassung und Zusammenstellung der f�r die Auswertungben�tigten Formeln

\subsection{Fachnummern}
Die Formeln f�r das Modell sind in den Theoretischen Aufgaben enthalten. Dies sind
die Formeln welche f�r die Auswertung der Stichproben dienen.

Durchschnitt oder arithmetischer Mittelwert
\[ \overline{x}=\frac{1}{N}\sum\limits_{i=1}^Nx_i\]

Standardabweichung der Einzelmessung oder Fehler der Einzelmessung
\[ s_x = \frac{1}{\sqrt{N-1}}\sqrt{\sum\limits_{i=1}^N(x_i-\overline{x})^2} \]

Standardabweichung der Durchschnitte oder Fehler des Mittelwerts
\[ s_{\overline{x}}= \frac{1}{\sqrt{N(N-1)}}\sqrt{\sum\limits_{i=1}^N(x_i-\overline{x})^2} \]

\subsection{Theoretische Aufgaben}
Dies sind die L�sungen zu den Aufgaben aus dem Skript\cite{Script} f�r die Vorbereitung.

\subsubsection{Aufgabe 1}
\[ \binom {n} {n-k}=\frac{n!}{(n-k)!(n-(n-k))!}=\frac{n!}{k!(n-k)!}=\binom{n}{k} \]

\subsubsection{Aufgabe 2}
Binomischer Lehrsatz (Beweis: Analysis I Skript\cite{Analysis})
\[ (x+y)^n=\sum\limits_{k=0}^n \binom{n}{k} x^k y^{n-k} \]
\[ x,y = 1  \Rightarrow  2^n = \sum\limits_{k=0}^n \binom{n}{k} \]


\subsubsection{Aufgabe 3}
\[ n\binom {n-1} {n-k}=\frac{(n-1)!}{(k-1)!((n-1)-(k-1))!}=\frac{n!}{(k-1)!(n-k)!}=k\frac{n!}{k!(n-k)!}=k\binom{n}{k} \]

\subsubsection{Aufgabe 4}
\[ \sum\limits_{x=0}^nP_n(x) = \sum\limits_{x=0}^n \binom{n}{x} p^k q^{n-x} = (p+q)^n\]

\subsubsection{Aufgabe 5}
\[ p=\frac{2}{3}, q=\frac{1}{3}, x=3, n=4 \]
\[ P_n(x) = \binom{4}{3}(\frac{2}{3})^3(\frac{1}{3})^1=\underline{\underline{0,395}} \]

\subsubsection{Aufgabe 6}
\[ p=\frac{1}{2}, q=\frac{1}{2}, x=3, n=8 \]
\[ 2^8P_n(x) = 2^8\binom{8}{3}(\frac{1}{2})^3(\frac{1}{2})^5=\underline{\underline{56}} \]

\subsubsection{Aufgabe 7}
\[ F(x)=50P_{14}(x)=50\binom{14}{x}(\frac{1}{2})^x(\frac{1}{2})^{14-x} \]
\begin{center}
\begin{tabular}{ll}
\textbf{Fach} & \textbf{F(x)} \\
 0, 14 & $3,1*10^{-3}$ \\
 1, 13 & $4,27*10^{-2}$ \\
 2, 12 & $2,77*10^{-1}$ \\ 
 3, 11 & 1,11 \\
 4, 10 & 3,05 \\
 5, 9  & 6,2 \\
 6, 8  & 9,16 \\
 7     & 10.47
\end{tabular}
\end{center}

\subsubsection{Aufgabe 8}
\[ \sigma_{\mu}= \sqrt{\frac{3,5}{N}} \]
a) 1,87 b) 0,26 c) 0,059

\subsection{Fehlerrechnung}

\subsubsection{Fachnummern}
Durchschnitt und Standardabweichung der Fachnummern

\[\overline{x}= \]

\subsection{Versuchsaufbau}
\begin{itemize}
\item Galton-Brett Nr. 6
\end{itemize}
% - Skizze
% - Ger�tenummern
% ...

\section{Messungen}
% - Alle Messwerte in �bersichtlichen Tabellen mit Mittelwerten und Fehlerangaben
\pgfplotstabletypeset[
  col sep=comma,
  every head row/.style={before row=\hline,after row=\hline\hline},
  every last row/.style={after row=\hline},
  every first column/.style={
    column type/.add={|}{}
  },
  every last column/.style={
    column type/.add={}{|}
  }
]{versuch_01_galtonbrett-messung_01.csv}

\section{Diskussion}
% - Kommentare
% - Vergleich der Messwerte mit Theorie und Literaturwerten
% - Hinweis auf m�glich Fehlerquellen (besonders systematische Fehler)
% - Schwierigkeiten bei der Messung
% ...

\section{Grafische Darstellungen}

\begin{thebibliography}{99}

\bibitem{Script} Anleitung zum Physikpraktikum, Praktikumsskript FS 2012, P. Wurz

\bibitem{Analysis} Skript zu Analysis I, C. Tretter

\end{thebibliography}

\end{document}