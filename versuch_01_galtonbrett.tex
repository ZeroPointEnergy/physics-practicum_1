\documentclass[english,ngerman]{article}
\usepackage[T1]{fontenc}
\usepackage[latin9]{inputenc}
\usepackage{geometry}
\geometry{verbose,tmargin=3.5cm,bmargin=3.5cm,lmargin=3cm,rmargin=3cm}
\setcounter{secnumdepth}{2}
\usepackage{float}
\usepackage{textcomp}
\usepackage{amstext}
\usepackage{amsmath}
\usepackage{graphicx}
\usepackage{esint}
\usepackage{babel}
\usepackage{pgfplotstable}

\makeatletter
\makeatother

\begin{document}

\title{Einf�hrung ins experimentelle Arbeiten}
\author{Andreas Zuber (03-186-905), Hans Muster (11-111-111)}
\date{24.2.2012}
\maketitle

\section{Einleitung}

\subsection{Ziel des Versuches}

\subsection{Literatur}
\begin{itemize}
\item Anleitung zum Physikpraktikum, Praktikumsskript FS 2012
\end{itemize}

\section{Theorie}
% - Zusammenfassung und Zusammenstellung der f�r die Auswertungben�tigten Formeln
\subsection{Aufgabe 1}
\[
\binom {n} {n-k}=\frac{n!}{(n-k)!(n-(n-k))!}=\frac{n!}{k!(n-k)!}=\binom{n}{k}
\]

\subsection{Aufgabe 2}
?

\subsection{Aufgabe 3}
\[
n\binom {n-1} {n-k}=\frac{(n-1)!}{(k-1)!((n-1)-(k-1))!}=\frac{n!}{(k-1)!(n-k)!}=k\frac{n!}{k!(n-k)!}=k\binom{n}{k}
\]


\subsection{Theoretische Aufgaben}

\subsection{Fehlerrechnung}

\subsection{Versuchsaufbau}
% - Skizze
% - Ger�tenummern
% ...

\section{Messungen}
% - Alle Messwerte in �bersichtlichen Tabellen mit Mittelwerten und Fehlerangaben
\pgfplotstabletypeset[
  col sep=comma,
  every head row/.style={before row=\hline,after row=\hline\hline},
  every last row/.style={after row=\hline},
  every first column/.style={
    column type/.add={|}{}
  },
  every last column/.style={
    column type/.add={}{|}
  }
]{versuch_01_galtonbrett-messung_01.csv}

\section{Diskussion}
% - Kommentare
% - Vergleich der Messwerte mit Theorie und Literaturwerten
% - Hinweis auf m�glich Fehlerquellen (besonders systematische Fehler)
% - Schwierigkeiten bei der Messung
% ...

\section{Grafische Darstellungen}

\end{document}